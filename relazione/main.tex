% !TeX spellcheck = it_IT
%Carattere dimensione 12
\documentclass[11pt]{article}

\usepackage[italian]{babel} % conversione in italiano
\usepackage{libertine}
\usepackage{graphicx}
\usepackage{amsmath}
\usepackage{pgfplots}
\usepackage{amssymb}
\usepackage{bm} % fornisce il comando \bm per il testo in grassetto
\usepackage[utf8]{inputenc}
\usepackage{float} % aggiunge l'opzione H alle "floating figures"
\usepackage{alphalph} % permette di generare lettere dato un indice numerico

\usepackage{standalone}
\usepackage{tikz}
\usepackage{caption, setspace}
\usepackage{hyperref}
\usepackage{import}
\usepackage{booktabs}
\usepackage{array} 
\usepackage{subcaption}
\usepackage{algorithmicx}
\usepackage[noend]{algpseudocode}


\pgfplotsset{width=11cm,compat=1.9}
\usepgfplotslibrary{external}
\usetikzlibrary{external}
\usetikzlibrary{positioning}
\usetikzlibrary{arrows}
\usetikzlibrary{calc}
\tikzexternalize%


\usepackage[top=2cm, bottom=2cm, left=2cm, right=2cm]{geometry}
\pagestyle{plain}
\linespread{1.0}
\captionsetup{margin=10pt,font={small},labelfont=bf} % caption styling

\newcommand{\includetikz}[2]{%
  \centering%
  \scalebox{#1}{\input{#2}}%
}

\begin{document}
\begin{titlepage}
  \begin{figure}[t]
    %\centering\includegraphics[width=0.9\textwidth]{marchio-unipi.eps}
    \centering\includegraphics[width=0.9\textwidth]{./img/logo.eps}
    
    \vspace{1cm}
    
    \centering\includegraphics[width=0.4\textwidth]{./img/cherubino.eps}
  \end{figure}

  \begin{center}
    \textbf{ Dipartimento di Informatica\\ Corso di Laurea Triennale in Informatica\\}
    \vspace{15mm}
    {\LARGE{\bf Relazione di accompagnamento per il progetto di Sistemi
        Operativi}}\\
  \end{center}

  \vspace{50mm}

  \begin{minipage}[t]{0.47\textwidth}
    {
    }
  \end{minipage}\hfill
  \begin{minipage}[t]{0.47\textwidth}\raggedleft
    {\LARGE{\bf Autore: \\ Marco Pampaloni}}
  \end{minipage}

  \vfill

  \centering{\large{\bf Anno Accademico 2019/2020 }}

\end{titlepage}

\section{Introduzione}
Il progetto descritto in questo elaborato è da intendersi come soluzione
proposta per il progetto finale semplificato relativo al corso di Sistemi
Operativi. La struttura del documento ricalca, per semplificare la trattazione,
quella del codice sorgente sviluppato. Il codice è stato sviluppato e testato
su una macchina Ubuntu 20.04 a 64 bit ed è conforme POSIX nella sua interezza,
non includendo altre dipendenze se non la libreria \emph{pthread}. Tutto il
codice è versionato e ad ogni iterazione è stato testato con valgrind in modo
da garantire l'assenza di memory leak. Inoltre sono stati implementati vari
test di unità eseguibili automaticamente tramite il comando \emph{make test}.
Ciò è descritto più in dettaglio nella sezione~\ref{sec:makefile}.
Infine, il codice è dettagliatamente commentato, sia nella specifica che
nell'implementazione. I commenti al codice sono complementare a questo
elaborato e descrivono precisamente il funzionamento delle singole funzioni,
oltre alla loro specifica formale.

\section{Makefile}\label{sec:makefile}
Il Makefile è stato scritto con in mente l'utilizzo attivo di tecniche di
testing automatico. Per questo, diverse funzionalità del programma sono state
testate creando dei file che testassero i requisiti dettati
dall'implementazione. Il Makefile gestisce la loro compilazione ed esecuzione,
oltre a controllare automaticamente l'output dei test, confrontandolo con
quello atteso. I test di unità sono poi ricchi di asserzioni interne volte a
verificare la correttezza dei post e prerequisiti.

Il Makefile presenta tre target di tipo phony: \emph{test, test2} e
\emph{clean}. Il target test esegue tutti i singoli test di unità e, solo se
terminano con successo, procede all'esecuzione del test di simulazione test2
prodotto secondo i termini della specifica di progetto. Infine il target clean
ripulisce la directory del progetto da tutti i file temporanei, file oggetto ed
eseguibili.

\section{Script di analisi}
Lo script Bash \emph{analisi.sh} parsa il file di log prodotto dalla
simulazione e fornisce in output (su di un file) e in bella forma le
informazioni raccolte durante l'esecuzione del programma. Lo script utilizza
prevalentemente i comandi \emph{cut, grep, sed} e \emph{awk} per formattare il
file di log e verificare alcune condizioni sui valori prodotti. Poichè la
specifica non era molto rigorosa, i test implementati sono vari e i valori
soglia sono arbitrari.

\section{Strutture dati e funzioni di utilità}
Per coadiuvare lo sviluppo e la modularità del codice, sono state implementate
varie strutture dati e funzioni di utilità, tra le quali una coda FIFO thread
safe (\emph{queue.h, queue.c}), un parser per il file di configurazione della
simulazione (\emph{parser.h, parser.c}), una threadpool per gestire la
creazione e l'esecuzione concorrente della clientela (\emph{threadpool.h,
  threadpool.c}), un cronometro utilizzato per misurare gli intervalli di tempo
impiegati dal programma per eseguire sotto-task o per fini statistici
(\emph{stopwatch.h, stopwatch.c}) e delle funzioni di logging thread safe per
scrivere su disco statistiche di utilizzo.

Ognuna delle strutture dati implementate fornisce anche le funzionalità
necessarie a liberare la memoria occupata dal loro utilizzo. Nel seguito
saranno descritte le strutture più importanti, lasciando la documentazione di
quelle minori all'implementazione stessa.

\subsection{Queue}
Il file queue.c implementa una coda FIFO unidirezionale e fornisce le
funzionalità di push, pop, top, size, empty e l'operazione funzionale map. La
coda è thread-safe nelle sue operazioni di base, le quali garantiscono
l'esecuzione atomica delle istruzioni contenute. Per questo motivo la coda è
sempre passata alle funzioni come puntatore (a dati non costanti) poichè il
lock è contenuto all'interno di essa ed è unico per ogni coda.

\subsection{Threapool}
Il file threadpool.c implementa una threadpool di dimensione fissa. Ad una
threadpool si possono sottomerre dei task, i quali saranno eseguiti da un
singolo thread non appena ve ne sarà uno disponibile. Inizialmente la
threadpool alloca il massimo numero di thread specificati al momento della
creazione e ognuno si mette in attesa di task su una coda condivisa e thread
safe.

La threadpool è utilizzata all'interno del progetto per gestire l'esecuzione
dei thread clienti e la loro liberazione dalla memoria. Un thread dedicato
all'interno dell'entry point del programma (\emph{simulazione.c}) si occupa
invece di gestire la creazione della threadpool e quella scaglionata di tutti i
clienti.

\subsection{Logger}
Il file logger.c implementa le funzionalità di logging utilizzate dai vari
thread della simulazione. La chiamata a \emph{log\_write()} comporta la
scrittura su disco del contenuto della stringa formattata passata come
parametro alla funzione. Ogni chiamata è thread safe e la scrittura concorrente
su file è garantita che sia completata prima di passare il controllo a un nuovo
thread. Il logger è inizializzato durante la creazione del supermercato e
deallocato solo dopo la sua chiusura.

\section{Supermercato}
Il file supermercato.c fornisce l'implementazione delle funzionalità richieste
per la simulazione di un generico supermercato. Il supermercato non è un attore
della simulazione, poichè non gli è fornito un thread indipendente su cui
svolgere il lavoro, ma è bensì una struttura dati utilizzabile tramite
l'interfaccia definita dal file supermercato.h. Tale interfaccia permette di
creare un supermercato, allocandolo dinamicamente sullo heap, aprire o chiudere
una cassa (la cassa è scelta internamente dall'implementazione), posizionare un
cliente in coda ad una cassa aperta (arbitrariamente scelta
dall'implementazione) e chiudere il supermercato, aspettando la terminazione di
tutte le casse ancora attive.
Alla creazione del supermercato, la quale richiede un record contenente i
parametri di configurazione, vengono inizializzati tutti i cassieri e creati i
thread dei singoli cassieri attivi, il cui numero è pari a quanto richiesto
dalla simulazione e fornito tramite il record di configurazione.
Alla chiusura (\emph{close\_supermercato()}) vengono processati i dati
statistici raccolti dai vari cassieri, aggregati e scritti sul file di log.
I log relativi ai singoli clienti sono invece prodotti durante la simulazione
dai thread clienti.

\section{Cassiere}
Il Cassiere è un attore della simulazione e gli è concesso un proprio thread di
esecuzione. Anche esso è implementato come una struttura dati conforme a una
interfaccia in stile API POSIX, definita nel file cassiere.h. Le funzionalità
proposte consentono di ottenere le informazioni di attività del cassiere (id,
stato di esecuzione, ecc.) e di causare l'apertura/chiusura delle casse.

Il thread del cassiere è implementato come un loop che termina quando viene
ricevuto un segnale di chiusura dall'esterno (viene settato lo stato interno
del cassiere). All'interno del ciclo di vita, il cassiere processa la sua coda
clienti, la quale è personale e protetta nell'accesso da una lock. Quando la
coda è vuota il cassiere si mette in attesa di nuovi clienti, per un periodo
lungo al più $S$ secondi. Quest'ultimo è un parametro di configurazione fornito
dall'utente e indica l'intervallo di tempo passato il quale, il cliente deve
comunicare al direttore il numero di clienti correntemente in coda alla cassa.
Poichè la specifica non era esaustiva sotto questo punto, è stato scelto di
rispettare questo intervallo cercando sempre di servire prima il cliente e,
qualora durante il tempo di servizio fosse stato necessario di comunicare con
il direttore, effettuare la comunicazione e tornare a servire il cliente. Ciò
permette di sfruttare al massimo le risorse del sistema e di non danneggiare il
tempo di servizio del cassiere. Il processamento del singolo cliente è simulato
con una chiamata alla funzione \emph{nanosleep()}.

Quando un cassiere termina, esso segnale la sua terminazione a tutti i clienti
attualmente in coda, i quali verranno ricollocati in altre casse aperte.

\section{Cliente}
Il Cliente è un altro attore della simulazione e la sua interfaccia è
specificata dal file cliente.h. Poichè i thread clienti sono numerosi e il loro
ciclo di vita teoricamente minimo, questi vengono gestiti tramite una
threadpool, la quale ricicla i thread degli utenti ormai terminati. La
dimensione della threadpool è pari al massimo numero di clienti che possono
essere presenti all'interno del supermercato.

Il cliente, dopo aver trascorso un tempo variabile in attesa per simulare la
scelta degli acquisti, si mette in coda ad una cassa si mette e attende di 
essere servito dal cassiere sulla sua variabile di condizione. Quando il
cassiere termina di processare il cliente, questo glielo segnala e il cliente
si risveglia, terminando la sua esecuzione e rilasciando le risorse alla
threadpool, la quale potrà fornirle ad altri task.

\section{Direttore}
Il thread direttore è l'ultimo attore della simulazione e la sua interfaccia è
specificata dal file direttore.h. Esso svolge sostanzialmente la verifica del
carico di lavoro sulle casse all'interno del supermercato. Durante il suo ciclo
di vita, il direttore controlla periodicamente i dati ricevuti dai cassieri
relativi ai clienti in coda e decide se aprire o meno le casse, in base a dei
valori soglia $S1$ e $S2$, forniti dal file di configurazione. La principale
aggiunta dell'implementazione riguarda un dettaglio non specificato dalla
richiesta del progetto, ovvero dopo quanto e con che frequenza il direttore
deve aprire o chiudere una cassa al verificarsi delle condizioni necessarie. Un
problema immediato lo si può verificare quando la condizione per l'apertura di
una cassa è verificata: il direttore apre conseguentemente una nuova cassa, se
disponibile e si rimette in attesa di nuovi eventi. Tuttavia le condizioni che
erano precedentemente verificate, rimangono tali per un intervallo di tempo
anche molto lungo. Ciò comporterebbe l'apertura quasi istantanea di tutte le
casse e lo stesso si verificherebbe per la loro chiusura, risultando in un
comportamento sicuramente non corretto.

Per ovviare al problema descritto, si è deciso di implementare una condizione
di ``pazienza'' per il direttore, il quale prima di prendere una decisione
riguardo l'apertura/chiusura di una cassa, aspetta che gli siano arrivate
un certo numero di comunicazioni da parte dei clienti. In tal modo dopo
l'apertura o la chiusura di una cassa, il supermercato ha il tempo di
stabilizzarsi e di dividersi il carico di lavoro tra le nuove casse aperte.

\section{Simulazione}
Il file simulazione.c è l'entry point del programma. Al suo avvio avviene la
registrazione dell'handler dei segnali POSIX, il quale gestisce i segnali
SIGHUP e SIGQUIT, come da specifica. L'handler non fa altro che settare
atomicamente lo stato interno della simulazione alla ricezione di un segnale,
causandone la terminazione dipendentemente dal valore dello stato.

Il passo successivo è il parsing del file di configurazione, cercato di default
nella directory locale, ma specificabile tramite argomento da linea di commando
tramite la sintassi \emph{./simulazione [-c configpath]}.

Viene poi creato il thread cha ha il compito di allocare i nuovi clienti e
sottomettere il loro thread alla threadpool. Il thread principale si mette poi
in attesa della ricezione di segnali e infine termina liberando tutte le
risorse allocate. Il thread di creazione dei clienti utilizza dei cleanup
handler per gestire la deallocazione delle risorse e attendere tutti i clienti
in caso di terminazione tramite segnale SIGHUP\@.


\section{Interdipendenze e analisi della concorrenza}
Il seguente grafico illustra le interazioni tra i vari attori della
simulazione, a cui vi è incluso il supermercato poichè durante l'esecuzione del
programma si comporta da tramite per le varie funzionalità.

Le linee tratteggiate descrivono chiamate di funzioni in cui il lock del
chiamante non è acquisito. Sono quindi chiamate safe che non possono
conseguire delle deadlock. Le linee spesse sono invece invocazioni con lock
acquisito. Il senso delle frecce stabilisce infine l'ordine tra chiamante e
chiamato, mentre l'assenza di cicli di linee piene mostra che non si possono
verificare deadlock (a patto che i lock siano gestiti correttamente).

\begin{figure}[h]
  \centering
  \begin{tikzpicture}[>=stealth]
    \node[circle, draw=black, minimum size=2.5cm] (direttore) at (5,10) {Direttore};
    \node[circle, draw=black, minimum size=2.5cm] (cliente) at (0,0) {Cliente};
    \node[circle, draw=black, minimum size=2.5cm] (cassiere) at (10,0) {Cassiere};
    \node[circle, draw=black, minimum size=2.5cm] (super) at (5,4.5) {Supermercato};

    % cassiere
    \draw[thick, ->] (cassiere) -- (direttore) %
    node[above=1pt, pos=0.5, scale=0.9, sloped] {comunica\_numero\_clienti()};

    \draw[densely dashed, ->] (cassiere.195) -- (cliente.345) %
    node[below=1pt, pos=0.5, scale=0.9, sloped] {set\_servito()};

    % cliente
    \draw[densely dashed, ->] (cliente) -- (direttore) %
    node[above=1pt, pos=0.5, scale=0.9, sloped] {get\_permesso()};

    \draw[thick, ->] (cliente.15) -- (cassiere.165) %
    node[above=1pt, pos=0.5, scale=0.9, sloped] {is\_cassa\_*()};

    \draw[thick, ->] (cliente.30) -- (super.210) %
    node[above=1pt, pos=0.5, scale=0.9, sloped] {place\_cliente()};

    % supermercato
    \draw[thick, ->] (super.345) -- (cassiere.135) %
    node[above=1pt, pos=0.5, scale=0.9, sloped] {is\_cassa\_*()};

    \draw[thick, ->] (super.325) -- (cassiere.155) %
    node[below=1pt, pos=0.5, scale=0.9, sloped] {add\_cliente()};
    
    % direttore
    \draw[densely dashed, ->] (direttore.south) -- (super.north) %
    node[above=1pt, pos=0.5, scale=0.9, sloped] {open\textbackslash close\_cassa()};

  \end{tikzpicture}
  \caption{Analisi grafica delle interdipendenze tra i vari attori della %
    simulazione}%
  \label{fig:dep}%
\end{figure}


\end{document}
