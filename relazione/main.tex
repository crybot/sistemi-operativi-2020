% !TeX spellcheck = it_IT
%Carattere dimensione 12
\documentclass[12pt]{report}

\usepackage[italian]{babel} % conversione in italiano
\usepackage{libertine}
\usepackage{graphicx}
\usepackage{amsmath}
\usepackage{pgfplots}
\usepackage{amssymb}
\usepackage{bm} % fornisce il comando \bm per il testo in grassetto
\usepackage[utf8]{inputenc}
\usepackage{float} % aggiunge l'opzione H alle "floating figures"
\usepackage{alphalph} % permette di generare lettere dato un indice numerico

\usepackage{standalone}
\usepackage{tikz}
\usepackage{caption, setspace}
\usepackage{hyperref}
\usepackage{import}
\usepackage{booktabs}
\usepackage{array} 
\usepackage{subcaption}
\usepackage{algorithmicx}
\usepackage[noend]{algpseudocode}
\usepackage[chapter]{algorithm}


\pgfplotsset{width=11cm,compat=1.9}
\usepgfplotslibrary{external}
\usetikzlibrary{external}
\usetikzlibrary{positioning}
\usetikzlibrary{arrows}
\usetikzlibrary{calc}
\tikzexternalize%


\usepackage[top=3cm, bottom=3cm, left=3cm, right=3cm]{geometry}
\pagestyle{plain}
\linespread{1.5}
\captionsetup{margin=10pt,font={small},labelfont=bf} % caption styling

\newcommand{\includetikz}[2]{%
  \centering%
  \scalebox{#1}{\input{#2}}%
}

\begin{document}
\begin{titlepage}
  \begin{figure}[t]
    %\centering\includegraphics[width=0.9\textwidth]{marchio-unipi.eps}
    \centering\includegraphics[width=0.9\textwidth]{./img/logo.eps}
    
    \vspace{1cm}
    
    \centering\includegraphics[width=0.4\textwidth]{./img/cherubino.eps}
  \end{figure}

  \begin{center}
    \textbf{ Dipartimento di Informatica\\ Corso di Laurea Triennale in Informatica\\}
    \vspace{15mm}
    {\LARGE{\bf Relazione di accompagnamento per il progetto di Sistemi
        Operativi}}\\
  \end{center}

  \vspace{20mm}

  \begin{minipage}[t]{0.47\textwidth}
    {
    }
  \end{minipage}\hfill
  \begin{minipage}[t]{0.47\textwidth}\raggedleft
    {\large{\bf Autore: \\ Marco Pampaloni}}
  \end{minipage}

  \vfill

  \centering{\large{\bf Anno Accademico 2019/2020 }}

\end{titlepage}

\end{document}
